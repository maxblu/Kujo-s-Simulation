\documentclass[]{article}


% Title Page
\title{Kojo's Kitchen}
\author{Daniel de la Osa Fernandez\\ c412}


\begin{document}
\maketitle



\begin{section} {An\'{a}lisis de Kojo´s Kitchen}
La cocina de Kojo es uno de los puestos de comida r\'{a}pida en un centro comercial. El centro comercial está abierto entre las 10:00 am y las 9:00 pm cada d\'{i}a. En este lugar se sirven dos tipos de productos: s\'{a}ndwiches y sushi. Para los objetivos de este proyecto se asumir\'{a} que existen s\'{o}lo dos tipos de consumidores: unos consumen s\'{o}lo s\'{a}ndwiches y los otros consumen s\'{o}lo productos de la gama del sushi. En Kojo hay dos per\'{i}odos de hora pico durante un d\'{i}a de trabajo; uno entre las 11:30 am y la 1:30 pm, y el otro entre las 5:00 pm y las 7:00 pm. El intervalo de tiempo entre el arribo de un consumidor y el de otro no es homog\'{e}neo, pero por conveniencia, se asumir\'{a} que es homog\'{e}neo. El intervalo de tiempo de los segmentos homog\'{e}neos, distribuyen de forma exponencial con el par\'{a}metro $\lambda$ tal que como media los intervalos de llegadas entre clientes son la siguiente: 

\begin{center}
	\begin{tabular}{|c|c|c|} \hline
		Per\'{i}odo 	&  Intervalo de llegadas   $\lambda$ (mins)	 \\ \hline
		10:00-11:30 	& 16	 						\\ \hline
		11:30-13:30 	& 2								\\ \hline	
		13:30-17:00 	& 13							\\ \hline
		17:00-19:00 	& 3		 						\\ \hline 	
		19:00-21:00		& 10	 						\\ \hline 
	\end{tabular}
\end{center}

Actualmente dos empleados trabajan todo el d\'{i}a preparando s\'{a}ndwiches y sushis para los consumidores. El tiempo de preparaci\'{o}n depende del producto en cuesti\'{o}n. Estos distribuyen de forma uniforme, en un rango de 3 a 5 minutos para la preparaci\'{o}n de s\'{a}ndwiches y entre 5 y 8 minutos para la preparaci\'{o}n de sushi.

El administrador de Kojo est\'{a} muy feliz con el negocio, pero ha estado recibiendo quejas de los consumidores por la demora de sus peticiones. \'{E}l est\'{a} interesado en explorar algunas opciones de distribución del personal para reducir el número de quejas. Su inter\'{e}s est\'{a} centrado en comparar la situaci\'{o}n actual con una opci\'{o}n alternativa donde un tercer empleado durante los per\'{i}odos m\'{a}s ocupados. La medida del desempe\~{n}o de estas opciones estar\'{a} dada por el por ciento de consumidores que espera m\'{a}s de 5 minutos por un servicio durante el curso de un d\'{i}a de trabajo. 

\end{section}


\begin{section} {Ideas seguidas para la soluci\'{o}n}
Este problema es muy similar al problema de atender a clientes con n servidores en paralelo con la diferencia de que todo el tiempo no est\'{a}n disponibles estos servidores, en nuestro problema los cocineros. Los clientes comenzaran a a llegar a la cafeter\'{i}a a pedir s\'{a}ndwich o sushi con siguiendo una distribuci\'{o}n exponencial como indica el ejercicio.

Cuando un cliente llega uno de los cocineros lo atiende lo que equivale a generar el tiempo de salida del cliente del establecimiento. Esto se logra generando una variable aleatoria con distribuci\'{o}n uniforme(a, b) en dependencia si lo que quiere es s\'{a}ndwich o sushi, para saber que es lo que quiere el cliente se genera un v.a con distriuci\'{o}n bernoulli con p=0,5. Esto se podría variar en dependencia si las personas piden mas un producto u otro. 

Otro caso ser\'{i} cuando el cliente llegue y no hay ning\'{u}n cocinero disponible en este caso se coloca en una cola y hasta que llegue el tiempo de salida de alguno de los que se están atendiendo y entonces se le da su tiempo de salida, siempre que sea el primero de la cola sino solo se acerca m\'{a}s a ser atendido pero sigue encolado. Se defini\'{o} el tiempo de espera del cliente como el tiempo en que sali\'{o} de la cola menos el tiempo en el que llego al establecimiento.

Sea T el tiempo en el que el establecimiento cierra, todos los clientes que est\'{e}n en cola ser\'{a}n atendidos y los clientes que lleguen pasada esa hora no se dejar\'{a} que entren; as\'{i} el sistema podr\'{a} atender a los que falta y despu\'{e}s terminar.

Para el an\'{a}lisis que se pide la idea es trabajar con la media del tiempo de demora de los clientes en varios d\'{i}as de simulaci\'{o}n de la cocina de Kujo. Esta cantidad de d\'{i}as la puede definir el usuario pero de no ser así se simulan 30 días inicialmente y luego se utiliza una f\'{o}rmula estad\'{i}stica para saber si con esa cantidad de d\'{i}as es suficiente para estimar el par\'{a}metro que se quiere con un error previamente definido en este caso se tomo error de 1 minuto. La f\'{o}rmula utilizada fue (formula). Donde d=1 S es la ra\'{i}z del estimador puntual para la varianza.    


                 


 
%Dentro de los elementos que hay que tener en cuenta es que los tiempos de los intervalos de llegadas de los clientes, que distribuyen de forma exponencial con parámetro lambda , este lambda fue tomado como 1/$\lambda$ para lograr tiempos en minutos mas realistas ya que la esperanza de la exponencial es 1/$\lambda$.                 

\begin{subsection} {Descripci\'{o}n del modelo}
\end{subsection}

\begin{subsection} {Variables}
\end{subsection}

\begin{subsection} {Inicialización }
\end{subsection}

\begin{subsection} {Seudocódigo}
\end{subsection}

\begin{subsection} {C\'{o}digo}

\begin{center}

\end{center}

\end{subsection}

\end{section}


\begin{section} {Aplicaci\'{o}n}
2 Aplicación
2.1 Vista Previa
2.1 Descripción
\end{section}

\begin{section} {Resultados}
\end{section}	
\end{document}
